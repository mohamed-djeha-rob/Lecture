\graphicspath{{Figures/}}
\chapter{Rappels sur les systèmes nonlinéaires} \label{chap:introduction_Gen}
\section{Rappels mathématiques}\label{sec-chap1:Rappel Math}
\begin{itemize}
	\item $\matriceA\inR^{n\times n}$ est une matrice définie positive si $\state^T\matriceA\state>0, \ \forall \state\inR^n-\{0\}$
	\item $\matriceA\inR^{n\times n}$ est une matrice définie positive si toutes ses valeurs propres sont positives
	\item $\matriceA\inR^{n\times n}$ est une matrice semi-définie positive si $\state^T\matriceA\state\geq0, \ \forall \state\inR^n-\{0\}$	
	\item $\matriceA\inR^{n\times n}$ est une matrice semi-définie positive si toutes ses valeurs propres sont positives ou nulles
	\item $\matriceA\inR^{n\times n}$ est une matrice définie négative si $-\matriceA$ est définie positive
	\item $\matriceA\inR^{n\times n}$ est une matrice semi-définie négative si $-\matriceA$ est semi-définie positive
\end{itemize}
\section{Systèmes linéaires et nonlinéaires}
\section{Point d'équilibre et point de fonctionnnement}
\section{Linéarisation}
\section{Notions de stabilité au sens de Lyapunov}
\section{Analyse de stablité}
\subsection{Première méthode de Lyapunov (méthode indirecte)}
\subsection{Deuxième méthode de Lyapunov (méthode directe)}
\section{Stabilisation d'un système nonlinéaire avec la méthode indirecte de Lyapunov}