\graphicspath{{Figures/}}
\chapter{Rappels sur les systèmes nonlinéaires} \label{chap:introduction_Gen}
\pagenumbering{arabic}
\section{Rappels mathématiques}\label{sec-chap1:Rappel Math}
\begin{itemize}
	\item $\matriceA\inR^{n\times n}$ est une matrice définie positive si $\state^T\matriceA\state>0, \ \forall \state\inR^n-\{0\}$
	\item $\matriceA\inR^{n\times n}$ est une matrice définie positive si toutes ses valeurs propres sont positives
	\item $\matriceA\inR^{n\times n}$ est une matrice semi-définie positive si $\state^T\matriceA\state\geq0, \ \forall \state\inR^n-\{0\}$	
	\item $\matriceA\inR^{n\times n}$ est une matrice semi-définie positive si toutes ses valeurs propres sont positives ou nulles
	\item $\matriceA\inR^{n\times n}$ est une matrice définie négative si $-\matriceA$ est définie positive
	\item $\matriceA\inR^{n\times n}$ est une matrice semi-définie négative si $-\matriceA$ est semi-définie positive
\end{itemize}
\section{Systèmes linéaires et nonlinéaires}
De manière générale, un système nonlinéaire s'écrit comme suit 
\begin{align}\label{eq-chap1:generic nnlinear sys}
	\begin{split}
		\dot{\state} &= f(t,\state,\command), t\geq0\\
		\out &= h(t,\state)
	\end{split},
\end{align}
tel que $\state\inR^{n}$ est l'état du système, $\command\inR^m$ est la commande, $\out\inR^p$ est la sortie du système, et $f$ représente la dynamique du système. Quand la dépendance par rapport au temps est explicite, ce système est dit \emph{non-autonome}. En contre partie, quand cette dépendance est implicite, et le système s'écrit sous la forme
\begin{align}\label{eq-chap1:autonomous system}
	\begin{split}
		\dot{\state} &= f(\state),\\
		\out &= h(\state)
	\end{split},
\end{align}
alors, il est dit \emph{autonome}. Ceci veut dire que la dynamique du système (ç.a.d, son évolution dans le temps) dépend complètement de l'état $\state$. Par contre, la forme~\eqref{eq-chap1:autonomous system} ne veut pas dire forcément que la commande $\command$ est nulle. En effet,  la forme~\eqref{eq-chap1:autonomous system} peut être aussi obtenue en mettant $\command = \gamma(\state)$ dans~\eqref{eq-chap1:generic nnlinear sys}.

Pour l'étude des systèmes autonomes, on s'intérésse aux points d'équilibre.
%Quand le système est linéaire, la forme~\eqref{eq-chap1:autonomous system} devient 
%\begin{align}\label{eq-chap1:linear system}
%	\begin{split}
%		\dot{\state} &= \matriceA\state+\matriceB\command,\\
%		\out &= \matriceC\state
%	\end{split},
%\end{align}
%où la linéarité est par rapport à l'état et aussi par rapport à la commande.


\section{Point d'équilibre}
Considérons le système autonome~\eqref{eq-chap1:autonomous system} où $f:\setD\rightarrow\mathbb{R}^n$ avec le domaine $\setD\subset\mathbb{R}^n$. $\stateEq\in\setD$ est appelé un point d'équilibre si $\stateDot=f(\stateEq)=0$. L'importance de l'étude des points d'équilibre prend tout son sens: le système ne va pas changer d'état dans le futur si il est au point d'équilibre. Pour trouver tous les points d'équilibre, il faut résoudre l'équation algébrique $f(\stateEq)=0$. L'une des caractéristiques des systèmes nonlinéaires est la multiplicité des points d'équilibre. 
\begin{example}\label{expl:pendule simple} 
	Soit le système suivant qui représente le modèle d'un pendule simple en présence de frottement
	\begin{align*}
		\begin{split}
			\dot{x}_1 &= x_2\\
			\dot{x}_2 &=-ax_2 -b\sin(x_1)
		\end{split}.
	\end{align*}
	Les points d'équilibre sont situés aux points $\stateEq=\begin{bmatrix}
	n\pi\\0
	\end{bmatrix}$, $n\in\mathbb{Z}$. Physiquement, ils représentent les deux positions verticales du pendule où le couple due à la force de gravité est nulle. Si le pendule est initialement à ces deux positions, il va y rester dans le futur.
\end{example}
Il est à noter que les points d'équilibre ne sont pas forcément des points isolés, mais peuvent être aussi des surfaces.
\begin{example}
Soit le système suivant 
\begin{align*}
	\begin{split}
		\dot{x}_1 &=x_1(x_1^2+x_2^2-1) \\ 
		\dot{x}_2 &=-x_2(-x_1^2-x_2^2+1)
	\end{split}.
\end{align*}
Les points d'équilibre sont l'origine $(0,0)$ et tout le cercle centré de rayon 1. 
\end{example}
Revenons à l'Exemple~\ref{expl:pendule simple}. Bienque, exactement aux deux points d'équilibre, le pendule maintient son état en restant fixe, sont comportement est complètement lorsqu'il est au voisinage de ces deux points. En effet, si le pendule est au voisinage de l'origine, il va y converger. Par contre, il ne pourra jamais revenir au point d'équilibre qui représente la position inversée. Ceci montre qu'un système nonlinéaire peut avoir un comportement différent suivant le point d'équilibre. Ceci nous motive à s'intérésser à la stabilité du point d'équilibre. Néanmoins, que peut-il se passer lorsque le système est au voisinage de son point d'équilibre ? Dans ce cas on s'intérésse à la stabilité du point d'équilibre. Par soucis de généralisation, les définitions de stabilité sont établies pour un point d'équilibre à l'origine $\stateEq=0$.  Si $\stateEq\neq0$, on peut toujour se ramener l'origine en  posant $\bar{\state} = \state - \stateEq$. En dérivant $\bar{\state}$, on obtient
\begin{equation}
	\bm\dot{\bar{\state}} = \stateDot = f(\bar{\state}+\stateEq) =g(\bar{\state}),  
\end{equation} 
où $g(0)=0$.
\subsection{Stabilité du point d'équilibre au sens de Lyapunov}
On suppose que le système~\eqref{eq-chap1:autonomous system} a un point d'équilibre à l'origine $\stateEq=0$.
\begin{definition}\label{def:Stability}\cite[Définition~4.1]{khalil2002NonLinearSystems}
	Le point d'équilibre $\stateEq$ du système~\eqref{eq-chap1:autonomous system} est 
	\begin{itemize}
		\item stable (ou Lyapunov stable) si pour chaque $\epsilon>0$ il existe $\delta=\delta(\epsilon)>0$ tel que
		\begin{equation}\label{eq-chap1:lyapunov stable}
			\norm{\state(0)}<\delta\Rightarrow\norm{\state(t)}<\epsilon.
		\end{equation}
		\item asymptotiquement stable si il est stable, et $\delta$ peut être choisi telque 
		\begin{equation}\label{eq-chap1:asymptotic stable}
			\norm{\state(0)}<\delta\Rightarrow\underset{t\rightarrow\infty}{\lim}\norm{\state(t)}=0.
		\end{equation}
		\item exponentiellement stable si il est stable, et ils existent $\alpha,\lambda>0$ tel que
		\begin{equation}\label{eq-chap1:exp stable}
			\norm{\state(t)}\leq\alpha\norm{\state(0)}e^{-\lambda t}.
		\end{equation}
	\end{itemize}
\end{definition}
Quand le point d'équilibre est Lyapunov stable celà veut dire simplement que les trajectoires d'état sont bornées et confinées dans une boule dans $\mathbb{R}^n$ de rayon $\epsilon$. C'est la forme la plus legère de stabilité. La stabilité est un peu plus fortifiée quand elle est asymptotique:  les trajectoires d'état convergent vers le point d'équilibre (l'origine) quand $t$ tend vers l'infini. Il est à noter que, malgré cette convergence, la stabilité au sens de lyapunov est nécéssaire pour que garantir que les trajectoires d'état restent bornées. Finalement, la stabilité exponentielle fortifie d'avantage la stabilité asymptotique en imposant un profil de convérgence exponentielle au trajectoire d'état vers le point d'équilibre.

Bien que la \cref{def:Stability} aide à comprendre la notion de stabilité au sens de Lyapunov, elle necéssite la résolution du système nonlinéaire~\eqref{eq-chap1:autonomous system}. Souvent, ceci n'est pas possible. Dans la \cref{sec:stability analysis}, on étudiera des méthodes dites de Lyapunov pour la caractérisation de la stabilité du point d'équilibre sans la nécessité de résoudre le système nonlinéaire~\eqref{eq-chap1:autonomous system}. 
\section{Analyse de stablité}\label{sec:stability analysis}
\subsection{Première méthode de Lyapunov (méthode indirecte)}
Soit le système nonlinéaire~\eqref{eq-chap1:autonomous system} ayant l'origine comme point d'équilibre. Soit $\Delta \state = \state-\stateEq$. Alors autour du point d'équilibre, le système~\eqref{eq-chap1:autonomous system} peut s'écrire
\begin{align}
\begin{split}
\stateDot &= f(\state) = f(\stateEq+\Delta\state)\\
&=f(\stateEq=\zeros) + \left.\begin{matrix}
\frac{\partial f}{\partial\state}
\end{matrix}\right|_{{\state = \zeros}}\Delta\state + \bm{o} \\
\Rightarrow\stateDot&\approx\left.\begin{matrix}
\frac{\partial f}{\partial\state}
\end{matrix}\right|_{{\state = \zeros}}\Delta\state = \matriceA\Delta\state 
\end{split}
\end{align} 
La première méthode de Lyapunov, aussi connue par la \emph{méthode indirecte de Lyapunov} est anoncé par le théorème suivant.
\begin{theoreme}\label{thm:indirect Lyapunov method}
	Soit le système nonlinéaire~\eqref{eq-chap1:autonomous system} ayant l'origine comme point d'équilibre, et soit $\matriceA =\left.\begin{matrix}
	\frac{\partial f}{\partial\state}
	\end{matrix}\right|_{{\state = \zeros}}$. Alors,
	\begin{enumerate}
		\item L'origine est localement asymptotiquement stable si toutes les valeurs propres de $\matriceA$ sont à partie réelle négative $\left(\Leftrightarrow \forall\lambda_i(\matriceA): {\rm Re}\lambda_i(\matriceA)<0\right)$.
		\item L'origine est instable si au moins une des valeurs propres de $\matriceA$ est à partie réelle positive $\left(\Leftrightarrow \exists\lambda_i(\matriceA): {\rm Re}\lambda_i(\matriceA)>0\right)$.
	\end{enumerate}
\end{theoreme}
La méthode indirecte de Lyapunov éffectue une  linéarisation du système autour du point d'équilibre, puis statue sur la stabilité du système nonlinéaire en se basant sur la stabilité du système linéarisé. Dans cette approche, il y un avantage majeur : l'analyse de stabilité d'un système linéaire est simple car il suffit d'analyser les valeurs propores de la matrice $\matriceA$. (\red{TODO: monter comment l'analyse de stabilité est performée pour les systèmes linéaires.})

Par contre, la stabilité asymptotique est seulement prouvée pour être \emph{locale}. 
La stabilité asymptotique locale signifie qu'il existe un ensemble $\setD\subset\mathbb{R}^n$ contenant l'origine, tel que si $\state(0)\in\setD\Rightarrow \state(t)\overset{t_\rightarrow\infty}{\longrightarrow}0$. 

Dans ce qui suit, on va étudier la méthode directe de Lyapunov qui ne se base pas sur une approximation du système nonlinéaire, mais en contre partie utilise des fonctions dite de Lyapunov. 
\subsection{Deuxième méthode de Lyapunov (méthode directe)}\label{subsec:lyapunov direct method}
Commenceons par un exemple introductif. Considérons un pendule simple dont l'équation dynamique est 
\begin{align}\label{eq-chap2:pendulum}
	\begin{split}
		\dot{x}_1 &= x_2 \\
		\dot{x}_2 &= -\frac{mgl}{J}\sin(x_1)-\frac{f}{J}x_2
	\end{split}
\end{align}
L'énergie totale du pendule est donnée par 
\begin{equation}
	E = \fracOneTwo Jx_2^2+mgl(1-\cos(x_1))
\end{equation}
la dérivée par rapport au temps de $E$ donne
\begin{align}\label{eq-chap2:pendulum Edot}
	\begin{split}
		\dot{E}&=Jx_2\dot{x}_2+mgl\sin(x_1)\dot{x}_1\\
		 	   &= Jx_2\left(-\frac{mgl}{J}\sin(x_1)-\frac{f}{J}x_2\right) + mgl\sin(x_1)x_2\\
		 	   &=-fx_2^2 \leq0
	\end{split}
\end{align} 
\cref{eq-chap2:pendulum Edot} signife que si initialement l'énergie du pendule non nulle ($E>0$), alors va diminuer au fil du temps car le pendule dissipe de l'énergie à travers les frottements. De plus, cette dissipation disparait seulement si le pendule est à l'arrêt. Infine, l'énergie du pendule atteindra sa valeur minimale ($E=0$) qui correspond à la position d'équilibre $x_1=x_2=0$. 

Cet exemple nous montre qu'on peut analyser le comportement du système en utilisant son énergie son avoir besoin de linéarisation, ni de résolution de l'équation dynamique nonlinéaire~\eqref{eq-chap2:pendulum}. De manière générale, Lyapunov a montré que l'on peut juste utiliser des fonctions scalaires $V(\state)$ (définies positives) qui ne représentent pas forcément l'énergie du système, mais qui y ressemblent : $V$  strictement positives sur $\setD$ sauf à l'origine où $V(\state=\zeros)=0$. 

\begin{definition}\label{def:FDP}
	\begin{itemize}
		\item $V(\state)$ est une fonction définie positive (FDP) sur $\setD$ ssi 
		\begin{itemize}
			\item $V(\state)>0, \ \forall \state\in\setD-\left\{0\right\}$, et
			\item $V(\state)=0$ seulement à l'origine.
		\end{itemize}
		\item $V(\state)$ est une fonction semi-définie positive (FSDP) sur $\setD$ ssi $V(\state)\geq0, \ \forall \state\in\setD$ (c.à.d., $V(\state)$ s'annule à l'origine et à d'autres points.).
		\item $V(\state)$ est une fonction définie négative (FDN) sur $\setD$ ssi  $-V(\state)$ est FDP.
		\item $V(\state)$ est une fonction semi-définie négative (FSDN) sur $\setD$ ssi  $-V(\state)$ est FSDP.
	\end{itemize}
\end{definition} 
Il est important de noter que la Définition~\ref{def:FDP} est toujours relative à un espace $\setD$. Par exemple, 
\begin{itemize}
	\item La fonction $V(\state) = x_1^2$ est FDP sur $\setD=\mathbb{R}$, mais elle est FSDP $\setD=\mathbb{R}^2$.
	\item La fonction $V(\state) = x_1^2 + x_2^2$ est FDP sur $\setD=\mathbb{R}^2$, mais elle est FSDP $\setD=\mathbb{R}^3$.
\end{itemize}
À ce stade, on est prêt à annoncer le deuxième Théorème de Lyapunov.
\begin{theoreme}\label{thm:direct lyapunov method}
	Soit le système~\eqref{eq-chap1:autonomous system} ayant l'origine comme point d'équilibre, et soit la fonction de Lyapunov $V(\state)$ définie positive sur $\setD\subset\mathbb{R}^n$. Alors, 
	\begin{enumerate}
		\item L'origine du système~\eqref{eq-chap1:autonomous system} est localement Lyapunov stable si $\dot{V}$ est FSDN sur $\setD$.
		\item L'origine du système~\eqref{eq-chap1:autonomous system} est localement asymptotiquement stable si $\dot{V}$ est FDN sur $\setD$.
		\item L'origine du système~\eqref{eq-chap1:autonomous system} est localement exponentiellement stable si $\exists\alpha,c_{{i=\left\{1,2,3\right\}}}>0:\ c_1\norm{\state}^\alpha\leq V(\state)\leq c_2\norm{\state}^\alpha$, et $ \dot{V}\leq c_3\norm{\state}^\alpha, \ \forall\state\in\setD$.
	\end{enumerate}
\end{theoreme}
Le Théorème~\ref{thm:direct lyapunov method} est aussi appelé la \emph{méthode directe de Lyapunov}. La stabilité est \emph{locale} car elle définie par rapport $ \setD$. Pour que les conclusions de stabilité dans le Théorème~\ref{thm:direct lyapunov method} devienne \emph{globale}\footnote{La stabilité globale ne concerne que les systèmes avec un seul point d'équilibre !}, il faut satisfaire deux conditions : 
\begin{enumerate}
	\item $\setD=\mathbb{R}$
	\item $V(\state)$ est radialement non-bornée. Autrement dit, $V(\state)\longrightarrow\infty$ si  $\norm{\state}\longrightarrow\infty$
\end{enumerate}
Autant la première condition est très intuitive, la second condition est peu claire. En effet, cette condition est posée pour éviter que $V(\state)$ décrémente pendant que $\norm{\state}$ devient non-bornée. Par exemple, prenons le système nonlinéaire suivant 
\begin{align}\label{eq-chap2:example of nonglobal stable system}
	\begin{split}
		\dot{x}_1 &=(x_2-1)x_1^3\\
		\dot{x}_2 &=-\frac{x_1^4}{(x_1^2+1)^2}-\frac{x_2}{x_2^2+1}
	\end{split},
\end{align}
ayant l'origine comme unique point d'équilibre. Considérons la fonction de Lyapunov candidate
\begin{equation}\label{eq-chap2:example of lyapunov func}
	V(\state) = x_2^2 + \frac{x_1^2}{x_1^2+1}
\end{equation}
qui est FDP sur $\mathbb{R}^2$. $\dot{V}$ est donnée par 
\begin{equation*}
	\dot{V} = -2\frac{x_1^4}{(x_1^2+1)^2}-2\frac{x_2^2}{x_2^2+1}
\end{equation*}
On pourrait conclure que le système est globalement asymptotiquement stable car $\dot{V}$ est FDN sur $\mathbb{R}^2$. Par contre, on peut montrer par simulation que le système~\eqref{eq-chap2:example of nonglobal stable system} est asymptotiquement stable pour la condition initiale $\state(0)^T=\begin{bmatrix}
1 & 1.9
\end{bmatrix}$, mais il est instable pour  $\state(0)^T=\begin{bmatrix}
1 & 2
\end{bmatrix}$. En effet, la stabilité globale n'est pas garantie car $V(\state)$ dans~\eqref{eq-chap2:example of lyapunov func} n'est pas radialement non-bornée ($V(\state)$ est bornée à 1 dans la direction de $x_1$). 

Aussi, il est à noter que le Théorème~\ref{thm:direct lyapunov method} fournit des conditions \emph{suffisantes} de stabilité. En effet, trouver une fonction $V(\state)$ tel que $\dot{V}\geq0$ sur $\setD$ n'implique en aucun cas que le système est instable. 
\begin{example}
	Considérons le système nonlinéaire suivant 
	\begin{align}\label{eq-chap2:expl nonlinear sys}
		\begin{split}
			\dot{x}_1 &=-x_1 + x_2^2\\
			\dot{x}_2 &=-x_2
		\end{split},
	\end{align}
	ayant l'origine comme unique point d'équilibre. Étudions la stabilité de l'origine par les fonctions candidates suivantes 
	\begin{align}		
			\label{eq-chap2:V1}	V_1(\state)&= \fracOneTwo(x_1^2 + x_2^2),\\
			\label{eq-chap2:V2}	V_2(\state)&= \fracOneTwo x_1^2 + (x_1+\fracOneTwo x_2^2)^2,\\
		\label{eq-chap2:V3}		V_3(\state)&= \fracOneTwo x_1^2 + (x_1+\fracOneTwo x_2^2)^2 + \fracOneTwo x_2^2	.	
	\end{align}
	Les dérivées de $V_1,V_2$ et $V_3$ donnent
	\begin{align}\label{eq-chap2:Vdot1}		
		\dot{V}_1&= -x_1^2 - x_2^2(1-x_1),\\
		\label{eq-chap2:Vdot2}	\dot{V}_2&= -3x_1^2,\\
	\label{eq-chap2:Vdot3}		\dot{V}_3&= -3x_1^2 -x_2^2	.	
	\end{align}
	$\dot{V}_1$ est FDN sur $\setD = \left\{\state\inR^2: x_1<1\right\}$. Alors en utilisant la fonction candidate $V_1(\state)$, on conclue que l'origine du système~\eqref{eq-chap2:expl nonlinear sys} est asymptotiquement stable sur $\setD$. 
	$\dot{V}_2$ est FSDN sur $\mathbb{R}^2$. Alors en utilisant la fonction candidate $V_2(\state)$ (qui est radialement non-bornée), on conclue que l'origine du système~\eqref{eq-chap2:expl nonlinear sys} Lyapunov stable sur $\mathbb{R}^2$. 
	$\dot{V}_3$ est FDN sur $\mathbb{R}^2$. Alors en utilisant la fonction candidate $V_3(\state)$ (qui est radialement non-bornée), on conclue que l'origine du système~\eqref{eq-chap2:expl nonlinear sys} est globalement asymptotiquement. 
\end{example}
Il est important de remarquer que, pour un seul système~\eqref{eq-chap2:expl nonlinear sys}, on a eu trois conclusions différentes  sur la stabilité de son origine mais qui ne sont pas contradictoires. Ceci montre la difficulté relative à l'application du Théorème~\ref{thm:direct lyapunov method} pour l'analyse de stabilité des système linéaire : la méthode directe de Lyapunov ne fournit pas une approache constructive pour l'obtention d'une fonction de Lyapunov. En revanche trouver une fonction de Lyapunov reste tout de même moins difficile que la résolution d'un système différentiel nonlinéaire.
\subsection{Principe de l'invariance}
Dans la pratique, le mieux que l'on peut faire souvent c'est trouver $V(\state)$ FDP dont la dérivée est seulement FSDN. Par contre, on peut avoir des évidences (par simulation ou expérimentation) que le système est asymptotiquement stable. À titre d'exemple, le système~\eqref{eq-chap2:expl nonlinear sys} est seulement prouvé d'être Lyapunov stable en utilisant $V_2(\state)$ dans~\eqref{eq-chap2:V2}, alors qu'en réalité il est asymptotiquement stable tel que montré par le biai de   $V_3(\state)$ dans~\eqref{eq-chap2:V3}. Dans ces cas, on peut  utiliser le principe de Lasalle pour montrer que la stabilité asymptotique du système. 
\begin{corolaire}\label{cor:Lasalle lemma}
	Considérons le système~\eqref{eq-chap1:autonomous system} ayant l'origine comme point d'équilibre, et supposant qu'il existe $V(\state)$ FDP sur $\setD\subset\mathbb{R}^n$ tel que $\dot{V}$ est FSDN sur $\setD$. Alors, l'origine est localement asymptotiquement stable si l'ensemble limite invariant\footnote{Définition d'un ensemble limite invariant : } ${\cal F}(\state)=\left\{\state\in\setD: \dot{V}=0\right\}$ se réduit à  ${\cal F}(\state)=\left\{\state\in\setD: \state=\zeros\right\}$. 
\end{corolaire}
Le Principe de Lasalle (\cref{cor:Lasalle lemma}, aussi appelé le principe d'invariance) signifie qu'aucune solution $\state(t)$ ne peut identiquement rester sur ${\cal F}(\state)$ sauf l'origine.
La stabilité asymptotique globale peut aussi être montrée en utilisant si $\setD=\mathbb{R}^n$ et $V(\state)$ est radialement non-bornée. 

On peut appliquer le principe de Lasalle pour montrer que le système~\eqref{eq-chap2:expl nonlinear sys} est asymptotiquement stable en utilisant $V_2(\state)$ dans~\eqref{eq-chap2:V2}. En effet, $\dot{V}_2 = -3x_1^2$, et donc l'ensemble limite invariant ${\cal F}(\state) = \left\{\state\inR^2:\dot{V}_2 = -3x_1^2=0\right\}$. Ceci implique que les trajectoires du système~\eqref{eq-chap2:expl nonlinear sys} sont confinées à l'axe $x_1=0$. Mais en utilisant la dynamique du système~\eqref{eq-chap2:expl nonlinear sys}, on peut montrer que  
\begin{equation*}
	x_1\equiv0 \Rightarrow \dot{x}_1\equiv0 \Rightarrow x_2\equiv0
\end{equation*}
Ce qui montre qu'aucune trajectoire $\state(t)$ ne peut rester sur l'axe $x_1=0$ sauf à l'origine. Alors, par vertu du principe de Lasalle, l'origine est globalement asymptotiquement stable.
\section{Linéarisation autour du point de fonctionnement}

\begin{itemize}
	\item linearisation autour du point d'équilibre.
\end{itemize}


\section{Stabilisation d'un système nonlinéaire avec la méthode indirecte de Lyapunov}
\subsection{Point de fonctionnement}