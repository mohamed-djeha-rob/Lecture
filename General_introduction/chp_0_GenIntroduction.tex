\chapter*{Introduction} \label{chap:introduction}

\pagenumbering{arabic}
\setcounter{page}{1}

\markboth{Introduction}{}
\addcontentsline{toc}{chapter}{Introduction}
\epigraph{‘‘If we knew what it was we were doing, it would not be called research, would it?’’}{\emph{Albert Einstein}}
%\emph{
%	
%	‘‘’’
%	
%}
\section*{Motivation}
Robots' introduction into industry and manufacturing began more than 40 years ago~\cite{edwards1984appliedErgonomics}. The primary purpose was to supersede the human worker in repetitive, exhaustive, and non-safe tasks while providing high precision task execution. Since then, the robots have become more complex (e.g. mobile, redundant...) with and increasing workspace and dexterity. Such ameliorations allow them to emerge into a wide range of domains, and perform successfully complex tasks. During the last COVID pandemic, some robotics solutions are proposed to limit virus transmission in healthcare workspaces~\cite{guang-zhong2020scienceRobotics,tavakoli2020ais}. This dazzling achievement owes its success to the effort spent by the scientific community to develop generic control techniques capable of exploiting the robot's redundancy to perform secondary tasks while accounting for system limitations, safety requirements, and constraints applied by the environment.

Designing task-space control schemes that regulate the task error~\cite{samson1991book} has been a breakthrough leveraging sensory feedback. Jacobian-null-space-projection has been the first paradigm developed in this context~\cite{liegeois1977tsmc,siciliano1991icar,deluca1991springer}. 
%The idea consists on performing a primary task while achieving at best a secondary task projected in the Jacobian null-space of the primary task enabling a perfect decoupling between the control objectives. 
Although this approach provided successful results~\cite{dietrich2015ijrr}, it had for long a major drawback: non-handling of unilateral constraints. These control objectives expressed in terms of inequality constraints are inherently related to (i) the task to be achieved (e.g., establishing contacts and sustain them within their frictional set); (ii) the robot environment (e.g., avoiding collision with nearby obstacles, other robots, or human operators); or (iii) to the robot capabilities (e.g., joint bounds, hardware limitations, etc.). %Several approaches has been proposed to mitigate this issue~\cite{liegeois1977tsmc,khatib1985icra,schwienbacher2011icra,dietrich2011icra,mansard2009tro}, but they are conservative, lead to sub-optimal solutions and lack of generality. 

Alternatively, formulating the control problem as an optimal Quadratic Program (QP) control paradigm has been proposed as a promising numerical solution for its natural handling of inequality constraints. The first implementation of QP was in Inverse-Kinematics (IK) (first-order dynamics, with the joint velocity as a decision variable). It consisted of performing simple task-space reaching motion with robotic manipulators while accounting for joint bounds and limitations on joint acceleration and torque~\cite{faverjon1987icra,cheng1994tra,park1998icra,zhang2004transactionsonSysManCyb2}. After that, Inverse-Dynamics (ID) QP  control (second-order dynamics, with the joint acceleration as decision variable) was formulated for highly redundant robots yielding the possibility of meeting multiple control objectives simultaneously. Multi-objective QP paradigm has shown great capabilities in reproducing human behaviors on graphical characters in multi-contact setting~\cite{abe2007siggraph,collette2007humanoids} as well as on simulated humanoid robots~\cite{bouyarmane2011iros,bouyarmane2012humanoids_b,saab2011icra,kanoun2011tro,salini2010springer} by sorting the control objectives via soft or strict hierarchy. Although QP has been successfully implemented for real-time application on robotic arms~\cite{decre2009icra,rubrecht2010iros,rubrecht2012autonRobot}, straightforward extension to highly redundant humanoid robots was not possible due to high computation-time. However, the progress in Central Processing Units (CPU) and dedicated QP solvers~\cite{escande2010icra,escande2014ijrr} enabled successful real-time reactive applications~\cite{herzog2014iros,herzog2016autonomousRobot,vaillant2014humanoids,vaillant2016springer,galloway2015ieeeAccess,ames2014tac}.  

Nevertheless, the first and real-world test that QP controllers had to undergo was the DARPA Robotic Challenge (DRC)\footnote{The Defense Advanced Research Projects Agency (DARPA) is a research and development agency of the United States Department of Defense responsible for the development of emerging technologies for use by the military. The DRC is a competition of robot systems and software teams vying to develop robots capable of assisting humans in responding to natural and artificial disasters \url{https://www.darpa.mil/}.}. Several teams participated in this competition with different humanoids robots where the latter had to perform a set of complex tasks: walking through difficult terrains, ladder climbing, manipulating objects like opening doors and turning valves, clearing a passage of debris, driving a car, getting into narrow passages, etc. Although the participating teams showed great performances in one or several tasks, many research teams reported encountering serious instability issues and QP failures when preparing and during the DRC, such as strong oscillations and jerky motion~\cite{feng2015journalOfFieldRobotics,johnson2015journalOfFieldRobotics,koolen2016ijhr,dedonato2017frontiers}. 
%Several explanations can be put forward on the reasons for these instabilities, to cite a few:
%\begin{itemize}
%	\item The gap between the perfectly-modeled simulation and the uncertain world in practice: the robot dynamics model is only an ideal representation of the real system. However, these models are often not exact (neglecting some dynamics) and uncertain (the dynamic model parameters are not precisely known);
%	\item Hardware robots specification: whereas most robots are torque-controlled in simulation, some robots require (for technical reasons) additional terms, for instance, to fight static friction. These considerations are generally handled at a joint-level without any modification on the high-level QP control side.  
%%	\item Robots mechanical design (backlash, flexibilities), sensors noise and communication delays. 
%\end{itemize}

In addition,  sudden QP failures occurred when the robot has to take a configuration where one or several kinematic constraints are reached (we refer to \emph{kinematic constraints} all the constraints that are expressed as inequality in distances or velocities).   The QP failure denotes its non-ability to find a solution often due to an empty feasible set. 	
Unfortunately, the above issues have never been debugged and rigorously addressed. Instead, only palliative solutions have been proposed and reported to mitigate those issues based on qualitative observations without mathematical proofs~\cite{hopkins2015icra,kuindersma2016autonomousRobot,koolen2016ijhr,johnson2015journalOfFieldRobotics}. 

In the context of aircraft manufacturing,~\cite{kheddar2019ram} demonstrated the capability of humanoid robots in accessing narrow, cumbersome, or confined spaces. Unfortunately, our QP controller experienced the same instabilities and failures in the course of the execution. These issues caught our interest because of their repeatability. Namely, we can easily reproduce them with the same set of tasks and constraints parameters omitting the possibility of a random factor effect. %, especially when the robot is in extreme configurations.

The second topic we put our interest in is the unification of observation and control. The observation is necessary to estimate the states that are not accessible to measurement but required by the control. Often, the observation process is exogenous w.r.t QP. Hence, unifying observation and control within a holistic QP formulation shall open new insights for formulating control scenarios that inherently incorporate a data flow between the control and observation. %  Our objective is to draw a strategy to unify control and observation (so far performed as two separated processes) leveraging QP control framework. 
\section*{Contribution}
Our first objective in this thesis is to study the instability (namely in closed-loop) and the QP failure encountered issues. This topic is itself composed of three subtopics that are open questions and are still not thoroughly addressed: (i) \emph{how to properly formulate kinematic constraints?}; (ii) \emph{how to guarantee the stability and robustness of the closed-loop QP control scheme?}; and (iii) \emph{how to ensure constraints compatibility, i.e. dealing with interactions among the constraints?} 
%\begin{itemize}
%	\item how to properly formulate kinematic constraints?
%	\item how to guarantee the stability and robustness of the closed-loop QP control scheme?
%	\item how to ensure constraints compatibility?
%\end{itemize}
The second objective consists of drawing a strategy for the unification of observation and control as two built-in tasks in QP. The four contributions in this thesis address these questions and are made explicit below.
%For the QP control community, these are open questions and still not thoroughly addressed. 

First, the kinematic constraints require special attention since they cannot be directly introduced to QP. Instead, they need to be formulated in terms of the QP decision variables to be enforced. This formulation has to ensure the constraint satisfaction forward in time, i.e., the \emph{forward invariance} of the set defined by the kinematic constraint. For IK-QP, this question has been largely addressed since~\cite{faverjon1987icra}. However, it is not straightforward for ID-QP. 
The scientific community has spent a great effort to address this question, mainly by extending the same approaches performed in IK-QP to the second-order ID-QP. Nevertheless, the proposed methods focused on one type of constraint: joint limits, collision avoidance, CoM constraints, etc. In addition, they did not perform well in closed-loop, leading to a discontinuous and jerky motion near the constraints' bounds. Also, the lack of grounded proofs for forward invariance makes these approaches less reliable. Our contribution consists first of showing that all the constraints considered separately in previous works belong to the same class of kinematic constraints. Then, we enforce these constraints in QP by proposing an Ordinary Differential Inequality (ODI)-based formulation that enables an exponential convergence profile to the boundary. In particular, we consider the case where the kinematic constraints are introduced online to QP. Hence, we propose an adaptive gains method to ensure constraint satisfaction for all the initial conditions. The approach is validated experimentally on HRP-4 humanoid robot performing motions where several kinematic constraints are reached simultaneously~\cite{djeha2020ral}. 
    
Second, we study the stability of the closed-loop QP control scheme in the case of kinematic-controlled robots. The latter denote the stiff robots having high gains joint-controllers with desired joint position or velocity as input commands. These robots are widely used in both research and industry. Their stiffness makes them robust to static friction and model's uncertainties and enables them to perform precise motions. However, their joint-dynamics is hardly known, preventing them from being modeled in the QP controller. Few research works considered the joint-dynamics effect on the closed-loop system stability~\cite{singletary2022csl,molnar2022ral}.  
By considering the case of 1 Degree of Freedom (DoF) kinematic-controlled robots controlled in joint-space, we show using linear systems control-theory how the closed-loop QP stability can be lost when the task or constraint gains are set to high values. Nevertheless, redundant robots have high DoF with different joint-dynamics and are controlled in task-space. Hence, we use the Lyapunov control theory to show that the instability is due to the non-robustness against the non-modeled joint-dynamics. Our second contribution consists of proposing robust formulations for the task and kinematic constraints based on integral feedback terms that guarantee both tasks' robust stability and set robust asymptotic stability. Our formulation only assumes that the kinematic-controlled robot is Input-to-State-Stable (ISS), largely guaranteed in practice. Experiments have been performed on fixed-base manipulator Panda, and floating-base robot HRP-4~\cite{djeha2022tro-submitted}.  

Third, constraints compatibility is tightly related to conflicts between constraints. We precisely focus on the compatibility between kinematic constraints and hardware limitations. In this case, a QP failure occurs if the required deceleration to stop at a joint bound is higher than what is strictly allowed by the hardware limitations. However, dealing with such an issue requires predicting the robot's evolution and performing anticipatory control actions. Rather than explicitly reasoning about when to start decelerating, our idea is to modify the QP task reference targets to satisfy both kinematic and hardware constraints. We implement a Model Predictive Controller (MPC) layer on top of whole-body QP. Based on the closed-loop dynamics of the tasks, MPC predicts the task and joint states and enforces the kinematic constraints and hardware limitations along a finite preview horizon. Then, it outputs a sequence of task optimal targets to be tracked by whole-body QP, yielding a constraints compatible motion. Numerical simulations have been conducted using Panda manipulator.

Finally, we build our observation and control unification strategy by considering these two processes as ‘tasks’ of different natures: the observation task decreases the estimation error, whereas the control task minimizes the tracking error. More concretely, we consider the case where the target of the control task cannot be fully measured. Thus, the observation task aims at constructing the full-state of the target and forwarding it to the former task (or possibly others). To integrate these two tasks in one compact multi-objective QP formulation, we propose the novel concept of interdependent tasks: the state of the observation task (target full-state) is the input target for the trajectory tracking task. To demonstrate our approach, we apply it to the formulation of human-robot handover control. In such a scenario, the location to exchange the object between the handover agents is not known in advance. In addition, we often have access only to the object pose, which is time-varying. Hence, controlling the robot to converge the object pose systematically results in lagging motion since it lacks anticipation. However, the observation task's estimation of the full object state in terms of pose, velocity, and acceleration yields a seamless and anticipatory robot motion toward the object without explicitly planning the trajectory or priorly agreement on where to handover. Handover experiments have been performed using one robotic arm Panda. In addition, simulations have been performed to show the handover with multi-robot configuration~\cite{djeha2022arxiv}.

\section*{Thesis Organization}
This thesis is organized as follows. We dedicate \cref{chap:background} to the necessary preliminaries and related research. Then, we propose our ODI kinematic constraint formulation in~\cref{chap:adaptive gains}. After that, we tackle the closed-loop stability topic by first showing the instability issue on a simple case-study in~\cref{chap:instable qp}, then we propose the robust formulation for the general case in~\cref{chap:robust qp}. The unification of control and observation for human-robot handover is discussed in~\cref{chap:handover qp}, and MPC-based approach for constraints compatibility is addressed in~\cref{chap:mpc ref gov}.  Finally, we conclude our work with a general conclusion in \cref{chap:conclusion} highlighting the outline of this thesis and the perspectives. We put the definition required for the self-consistency of the thesis in~\cref{annex}, and the mathematical proofs in~\cref{appendix}.
%This thesis originated from the fact that the cited issues above have one interesting feature: they are extremely repeatable. Namely, we can easily reproduce them with the same set of tasks and constraints parameters. This fact systematically omits the possibility of a random factor effect. 
%Consequently, the first objective of this thesis is to address the following points: how to formulate the kinematic constraints 
%\begin{itemize}
%%	\item Explain what is the issue with kinematic constraints, online introduction
%%	\item Instability is a sign of non-robustness
%%	\item Constraints compatibility with MPC: we do not compute the \emph{right} moment to introduce the kinematic constraint, but we choose to modify the task and joint space reference targets if avoiding the constraints require it.  
%\end{itemize}