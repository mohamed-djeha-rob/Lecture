\chapter*{R\'esum\'e de la th\`ese}
La première utilisation de la programmation quadratique (QP) dans le contrôle robotique remonte aux années 1990. Il s'agissait d'une alternative aux méthodes basées sur la projection dans le noyau de la Jacobienne pour résoudre les redondances tout en tenant compte des contraintes unilatérales. Depuis lors, la commande QP est devenue un outil approprié pour combiner plusieurs objectifs de commande avec une hiérarchie souple ou stricte. Néanmoins, les contrôleurs QP ont encore quelques limitations et des questions ouvertes. Dans cette thèse, notre objectif est double : (i) aborder un certain nombre de problèmes ouverts du contrôle QP, et (ii) unifier le contrôle et l'observation via le contrôle QP multi-objectif.

Les contraintes cinématiques constituent une large classe de contraintes unilatérales qui ne peuvent pas être directement incluses directement dans la QP. Plusieurs solutions existent pour écrire ces contraintes en termes de variables de décision QP. Pourtant, aucune d'entre elles n'a donné de bons résultats en boucle fermée en raison de leur spécificité aux limites des articulations, de leur non-robustesse face aux dynamiques non modélisées et de leur manque de fondement théorique. Nous abordons ce sujet en proposant une formulation générale englobant toutes les contraintes cinématiques. Notre solution est basée sur l'inégalité différentielle ordinaire à gains adaptatifs avec des preuves formelles du respect des contraintes dans le temps.

Nous étudions également la stabilité du schéma de contrôle QP en boucle fermée pour les robots contrôlés en cinématique, c'est-à-dire les robots avec des contrôleurs d'articulation à gain élevé ayant la position ou la vitesse articulaire désirée comme consigne d'entrée. Bien que ces robots soient largement utilisés, le sujet de la stabilité n'a pas été exploré. En utilisant un système simple à un degré de liberté, nous montrons comment le schéma de contrôle en boucle fermée est sujet à l'instabilité, surtout si les gains de la tâche et/ou des contraintes sont fixés à des valeurs élevées. Ensuite, nous proposons une formulation robuste de la tâche et de la contrainte basée sur des termes de rétroaction intégrale qui donnent une stabilité robuste des tâches et une stabilité asymptotique robuste de l'ensemble défini par la contrainte cinématique. Notre solution s'applique à tout robot à commande cinématique sous des hypothèses pratiques.

Ensuite, nous abordons le sujet de la compatibilité des contraintes. Les contraintes sont incompatibles si elles sont en conflit. 
Par exemple, si la décélération requise pour arrêter le bras du robot avant d'entrer en collision est supérieure à ce qui est actuellement autorisé en raison d'autres contraintes. La gestion des conflits potentiels est typiquement une tâche d'anticipation qui nécessite une prédiction des mouvements. Notre solution consiste à mettre en œuvre un contrôleur prédictif basé sur le modèle (MPC) comme une couche au-dessus du QP du corps entier, et auquel nous déléguons la tâche de la compatibilité des contraintes. Le modèle MPC est construit sur la base de la dynamique des tâches et des contraintes cinématique en boucle fermée. En tenant compte des limites physiques et des contraintes cinématiques sur un horizon temporel fini, le MPC produit une séquence de consignes optimales suivies par le contrôleur QP, produisant un mouvement qui satisfait toutes les contraintes. 

Enfin, nous exploitons le paradigme du contrôle multi-objectif pour unifier deux tâches de nature différente : l'observation (estimation) et le contrôle (suivi). Nous formulons cette unification à travers le concept de tâches interdépendantes : l'état de la tâche d'observation est transmis comme une consigne pour la tâche de suivi, toujours via un contrôleur QP compact. Cette nouvelle formulation permet de générer un mouvement vers une cible observée. Typiquement, dans les scénarios de transfert humain-robot, l'emplacement du transfert n'est souvent pas connu à l'avance. Par conséquent, nous formulons un contrôleur QP pour le transfert d'objects bidirectionnel entre l'humain et le une robot dans l'espace des tâches où l'état complet de l'objet en termes de pose, de vitesse et d'accélération est observé et transmis comme référence pour la tâche de suivi. Notre formulation permet un mouvement continu et proactif du robot vers l'objet sans planification hors ligne ou connaissance préalable du lieu de rencontre.


%La première utilisation de la programmation quadratique (QP) dans le contexte du contrôle robotique remonte au début des années 90. Elle a été présentée comme une alternative prometteuse aux méthodes basées sur la projection dans l'espace nul de la Jacobienne afin de résoudre les redondances tout en tenant compte des contraintes unilatérales.  Depuis lors, le QP est devenu la norme en matière de contrôle de l'espace des tâches et s'est révélé l'outil approprié pour combiner et unifier différents objectifs de contrôle avec des schémas de hiérarchisation souples ou stricts. Cependant, les contrôleurs QP présentent encore certaines limitation et des questions ouvertes qui sont connues dans la communauté robotique. En particulier, les contraintes cinématiques constituent une grande classe de contraintes unilatérales qui ne peuvent pas être directement introduites dans le QP. Plusieurs solutions ont été proposées pour écrire ces contraintes en termes de variables de décision du QP. Cependant, aucune d'entre elles n'a donné de bons résultats en boucle fermée en raison de leur non-robustesse face aux dynamiques non modélisées, de leur manque de fondement théorique et de leur spécificité aux limites conjointes. Dans un premier temps, nous abordons ce sujet en proposant une formulation générale qui englobe toutes les contraintes cinématiques. Notre solution est basée sur les inégalités différentielles ordinaires à gains adaptatifs ce qui nous permet de prouver formellement la satisfaction des contraintes dans le temps même si elles sont introduites à la volée dans le QP.   
%
%Ensuite, nous abordons le sujet de la stabilité du contrôle QP en boucle fermée lorsque le robot est contrôlé en position, c'est-à-dire les robots avec des contrôleurs de bas-niveau à gains élevés avec une position ou une vitesse articulaire désirée comme consigne. En utilisant un système simple à un degré de liberté, nous montrons comment le schéma de contrôle en boucle fermée est sujet à l'instabilité, surtout si les gains des tâches et/ou des contraintes sont fixés à des valeurs élevées. Cette instabilité est due à la non robustesse contre les dynamiques non modélisées telles que la dynamique de bas-niveau, les flexibilités et les perturbations externes. Ensuite, nous considérons le cas général d'un robot contrôlé en position, et nous proposons une formulation robuste des tâches et des contraintes basée sur le feedback des termes d'intégrales qui donnent une stabilité robuste des tâches et une stabilité asymptotique robuste de l'ensemble défini par les contraintes cinématiques. La solution que nous proposons est destinée à être appliquée à tous les robots contrôlés en position sous des hypothèses pratiques.  
%
%Ensuite, nous abordons le sujet de la compatibilité des contraintes. Les contraintes sont incompatibles si elles sont en conflit dans le sens où elles ne peuvent pas être satisfaites simultanément. Dans un tel cas, l'ensemble de contraintes devient vide et QP ne peut pas trouver de solution. Par exemple, si la décélération requise pour arrêter le bras du robot avant d'entrer en collision est supérieure à ce qui peut être fourni par la limitation physique des moteurs. La gestion des conflits potentiels est typiquement une tâche d'anticipation qui nécessite la prévisualisation des états du robot. Par conséquent, notre solution consiste à mettre en œuvre un contrôleur prédictif en amont du QP. Le modèle du contrôleur prédictif est construit sur la  base des dynamique en boucle fermée des tâches  et  des contraintes cinématiques. En tenant compte des limitations physiques et des contraintes cinématiques sur un horizon fini, le contrôleur prédictif calcule les cibles optimales des tâches suivies par QP et qui convergent vers les consignes de référence tout en satisfaisant toutes les contraintes.   
%
%Enfin, nous exploitons la formulation du contrôle multi-objectif pour unifier l'observation et le contrôle. Nous proposons l'idée de tâches interdépendantes où l'état d'une tâche est transmis comme consigne de référence pour une autre tâche. Cela permet un mouvement proactif du robot vers un emplacement donné sans être défini préalablement.  Dans les scénarios de transfert d'objets humain-robot, l'emplacement du transfert n'est souvent pas connu à l'avance. C'est pourquoi, nous exploitons l'idée des tâches interdépendantes pour proposer une nouvelle formulation du transfer d'object human-robot dans l'espace des tâches. Nous proposons une formulation de l'espace des tâches du transfert humain-robot. Nous supposons seulement que la pose de l'objet est mesurée par un capteur. L'état complet de l'objet en termes de pose, de vitesse et d'accélération est alors construit par la tâche d'observation. Les états estimés du robot sont ensuite suivis par une tâche de suivi de trajectoire, ce qui permet un mouvement fluide et continu du robot vers l'objet sans aucune planification hors ligne ou connaissance préalable du lieu de rencontre.





