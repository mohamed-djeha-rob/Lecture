\graphicspath{{Figures/}}
\chapter{Commande Nonlinéaire par Backstepping} \label{chap:backstepping}
Dans la~\cref{subsec:lyapunov direct method}, on a vu comment analyser la stabilité d'un système nonlinéaire autonome en utilisant les fonctions de Lyapunov. Maintenant, supposons que l'on a un système nonlinéaire sous la forme~\eqref{eq-chap1:generic nnlinear sys}, où $\command$ est l'entrée de commande par laquelle on veut stabiliser le système. Le but est de trouver $\command$ telque le système~\eqref{eq-chap1:generic nnlinear sys} est asymptotiquement stable en boucle fermée. \emph{Comment peut-on exploiter la méthode directe de Lyapunov pour le problème de stabilization ?} 
\section{Commande nonlinéaire par les fonctions de Lyapunov} 
Exploiter une fonctions  de Lyapunov pour la commande consiste à effectuer les étapes suivantes : 
\begin{enumerate}
	\item Choisir $V(\state)$ FDP sur $\setD\subset\mathbb{R}^n$
	\item Calculer $\dot{V} = \frac{\partial V}{\partial \state}\stateDot = \frac{\partial V}{\partial \state}f(\state,\command)$
	\item Trouver $\command=\gamma(\state)$ tel que $\dot{V}$ est FDN sur $\setD$. 
\end{enumerate}
\begin{example}
	Soit le système nonlinéaire suivant
	 \begin{equation*}
		\dot{x} = x^5 + u
		\end{equation*}
	En choissant $V(x) = \fracOneTwo x^2$, on a $\dot{V} = x(x^5+u)$. Alors, si 
	\begin{equation}\label{eq-chap3:nnlinear feedback 1}
		u = -x^5-\lambda x, \  \lambda>0,
	\end{equation} alors $\dot{V} = -\lambda x^2$. 
\end{example}
On remarque que la commande dans~\eqref{eq-chap3:nnlinear feedback 1} est un retour d'état nonlinéaire à cause de la présence du terme $-x^5$. on va appliquer la même approche dans l'exemple suivant. 
\begin{example}
	Soit le système nonlinéaire suivant 
	\begin{align*}
		\dot{x}_1 &= x_1^2 + x_2 \\
		\dot{x}_2 &= x_1x_2 + (x_2^2+1)u
	\end{align*}
	On choisit la fonction de Lyapunov $V(\state) = \fracOneTwo(x_1^2 + x_2^2)$. Sa dérivée est donnée par 
	\begin{equation*}
		\dot{V} = x_1(x_1^2 + x_2 + x_2^2) + x_2(x_2^2 +1)u
	\end{equation*}
	Si on  choisit  u tel que 
	\begin{equation}\label{eq-chap3:nnlinear feedback 2}
		u=\frac{1}{x_2(x_2^2+1)}\left(-x_1(x_1^2 + x_2 + x_2^2)  - \lambda_1x_1^2 -\lambda_2x_2^2\right),
	\end{equation}
	alors, $\dot{V} = - \lambda_1x_1^2 -\lambda_2x_2^2$ FDN sur $\mathbb{R}^2$. 
\end{example}
Similairement à~\eqref{eq-chap3:nnlinear feedback 1}, u~\eqref{eq-chap3:nnlinear feedback 2} est aussi un retour d'état nonlinéaire. Par contre, u~\eqref{eq-chap3:nnlinear feedback 2} est indéfinie sur l'axe $x_2=0$. Ceci  renvoie à un \emph{mauvais choix} de la fonction de Lyapunov. Ce problème a été déja abordé dans la~\cref{subsec:lyapunov direct method} quant au choix des fonctions de Lyapunov pour l'analyse de stabilité. Dans ce qui suit, on va voir une technique de commande nonlinéaire pour construire de manière systématique une fonction de Lyapunov de tel manière que la commande est bien définie.  
\section{Méthode de Backstepping}
La méthode de backstepping est technique de commande nonlinéaire pour construire de manière systèmatique une fonction de Lyapunov pour la commande. Cette technique consiste à fragmanter le système  global en des sous-systèmes en cascade, puis de résoudre le problème de commande de manière recursive en partant du sous-système le plus interne (ordre 1), puis en remontant \emph{par étape} vers les ordres superieurs du système, d'où son nom \emph{Backstepping}. Néanmoins pour pouvoir appliquer cette technique, il faut que  le système nonlinéaire possède une structure trangulaire
\begin{align}\label{eq-chap3:strict-feedback form}
	\begin{split}
		\dot{x}_1 &= f_1(x_1) + g_1(x_1)x_2\\
		\dot{x}_2 &= f_2(x_1,x_2) + g_2(x_1,x_2)x_3\\
		&\vdots \\
		\dot{x}_{n-1} &= f_{n-1}(x_1,x_2,\ldots,x_{n-1}) +  g_{n-1}(x_1,x_2,\ldots,x_{n-1})x_n\\
		\dot{x}_n &= f_{n}(\state) +  g_{n}(\state)u
	\end{split},
\end{align} 
avec $g_i(x_1,x_2,\ldots,x_i)\neq0, \forall \state\in\setD$. Le système~\eqref{eq-chap3:strict-feedback form} a une forme dite en "retour d'état stricte", car pour chaque sous-système $i$ on retrouve seulement le retour des états $(x_1,x_2,\ldots,x_i)$. De plus,  la forme en cascade est visible dans~\eqref{eq-chap3:strict-feedback form} : 
\begin{itemize}
	\item $x_2$ apparait comme une entreé de commande \emph{fictive} pour le sous-système 1,
	\item $x_3$ apparait comme une entreé de commande \emph{fictive} pour le sous-système 2, $\cdots$
	\item  $x_n$ apparait comme une entreé de commande \emph{fictive} pour le sous-système $n-1$, 
	\item  $u$ apparait comme une entreé de commande \emph{réel} pour le sous-système $n$.
\end{itemize}

\section{Synthèse de la commande par Backstepping}
Pour expliquer la synthèse de la commande par Backstepping, considérons un système d'ordre 2 de la forme de~\eqref{eq-chap3:strict-feedback form}
%\begin{align}
	\begin{subequations}\label{eq-chap3:sys}
		\begin{align}
			\label{eq-chap3:subsys1}	\dot{x}_1 &= f_1(x_1) + g_1(x_1)x_2,\\
			\label{eq-chap3:subsys2}	\dot{x}_2 &= f_2(\state) + g_2(\state)u,
		\end{align}
%		\begin{equation}
%			\dot{x}_1 = f_1(x_1) + g_1(x_1)x_2
%		\end{equation}
%		\begin{equation}\label{eq-chap3:subsys2}
%			\dot{x}_2 = f_2(x_1,x_2) + g_2(x_1,x_2)u
%		\end{equation}
	\end{subequations}
%\end{align} 
avec $g_1(x_1),g_2(\state)\neq0$. On suppose qu'il existe une fonction $\phi_1(x_1)$ avec $\phi_1(0)=0$ tel que si $x_2 = \phi_1(x_1)$, alors le sous-système~\eqref{eq-chap3:subsys1} est asymptotiquement stable. De plus, on suppose que l'on connait $V_1(x_1)$ FDP sur $\mathbb{R}$ tel que 
\begin{equation}
	\dot{V}_1=\frac{\partial V_1}{\partial x_1}\left(f_1(x_1) + g_1(x_1)\phi_1(x_1)\right)\leq - W(x_1)
\end{equation}
tel que $W(x_1)$ est FDP sur $\mathbb{R}$. En posant 
\begin{align*}
	z&=x_2-\phi_1(x_1),\\
	u &= \frac{1}{g_2(\state)}\left(-f_2(\state) +\dot{\phi}_1(x_1) + v\right),
\end{align*}
  le système~\eqref{eq-chap3:sys} peut être ré-écrit comme suit 
	\begin{subequations}\label{eq-chap3:sys_modif}
	\begin{align}
			\dot{x}_1 &= \left(f_1(x_1) + g_1(x_1)\phi_1(x_1)\right) + g_1(x_1)z\\
			\dot{z} &= v
	\end{align}
	\end{subequations}
Finalement, il faut trouver la commande $v$ qui stabilize  $(x_1,z)$ à l'origine\footnote{$z\overset{t\rightarrow\infty}{\longrightarrow}0$ est un problème de poursuite. En effet, ceci est équivalent à dire $ x_2\overset{t\rightarrow\infty}{\longrightarrow}\phi_1(x_1)$.}. Il suffit de pose la fonction de Lyapunov $V_2$ suivante 
\begin{equation}
	V_2(x_1,z)=V_1 + \fracOneTwo z^2,
\end{equation}	
dont la dérivée $\dot{V}_2$ est 
\begin{align}\label{eq-chap3:V2 dot}
	\begin{split}
		\dot{V}_2 &= \frac{\partial V_1}{\partial x_1}\left(f_1(x_1) + g_1(x_1)\phi_1(x_1) + g_1(x_1)z\right) + zv \\ 
		&= \frac{\partial V_1}{\partial x_1}\left(f_1(x_1) + g_1(x_1)\phi_1(x_1)\right) + \left(v + \frac{\partial V_1}{\partial x_1}g_1(x_1)\right)z \\
		&\leq -W(x_1) + \left(v + \frac{\partial V_1}{\partial x_1}g_1(x_1)\right)z
	\end{split}
\end{align}
Alors, il est simple de voir qu'il suffit de choisir 
\begin{equation}\label{eq-chap3:integrator input backstepping}
	v= -\frac{\partial V_1}{\partial x_1}g_1(x_1) - \lambda_2z,
\end{equation}
pour que~\eqref{eq-chap3:V2 dot} soit FDN sur $\mathbb{R}^2$. Par la vértue du Théorème~\ref{thm:direct lyapunov method}, la commande 
\begin{equation}\label{eq-chap3:backstepping command}
	u = \frac{1}{g_2(\state)}\left(-f_2(\state) +\dot{\phi}_1(x_1) -\frac{\partial V_1}{\partial x_1}g_1(x_1) - \lambda_2\left(x_2-\phi_1(x_1)\right)\right)
\end{equation}
\subsection{Discussion}
La fonction de Lyapunov $V_2$ est en fonction de $x_1$ et $z$ pour s'assurer que la commande $v$ stabilise ces deux états à l'origine. De plus, il faut noter que celà assure la convergence de $x_2$ vers 0. En effet, $z\overset{t\rightarrow\infty}{\longrightarrow}0$ implique que $ x_2\overset{t\rightarrow\infty}{\longrightarrow}\phi_1(x_1)\overset{t\rightarrow\infty}{\longrightarrow}0$ par conséquence de la convergence de $x_1$ vers 0. 
Aussi, la commande par backstepping $u$~\eqref{eq-chap3:backstepping command} est non-unique par effet de la non-unicité du choix de $V_1$. Finalement, la synthèse de la commande par backstepping comme expliquée  peut être directement généralisé pour des systèmes d'ordre supérieur comme le montre l'exemple suivant. 
\begin{example}
	Soit le système nonlinéaire de 3ème ordre 
	\begin{align}
		\begin{split}
			\dot{x}_1&= x_1^2 + (x_1^2+1)x_2 \\
			\dot{x}_2&= x_1x_2 + x_2\\
			\dot{x}_3&= x_2 + u 
		\end{split}
	\end{align}
\end{example}

